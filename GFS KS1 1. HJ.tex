\documentclass{beamer}
%\setbeamertemplate{navigation symbols}{}  %remove navigation symbols
%Deutsch
\usepackage[ngerman]{babel}
\usepackage[utf8]{inputenc}
%Mathe
\usepackage{amsmath}
%Einheiten
\usepackage{siunitx}
%Baumstrukturen
\usepackage{tikz}
\usepackage{tikz-qtree}
\usepackage{textcomp}   % allows \textrightarrow
%Bildquellen
\usepackage{varwidth}
\usepackage{graphicx}
\usepackage{hyperref}

\author{Florian Seligmann}
\title{Modelle in Chemie}
\date{\textbf{12.12.2017}}

\tikzset{
  invisible/.style={opacity=0},
  visible on/.style={alt={#1{}{invisible}}},
  alt/.code args={<#1>#2#3}{%
    \alt<#1>{\pgfkeysalso{#2}}{\pgfkeysalso{#3}}
  }
}

%Item mit Pfeil
\newcommand{\aitem}{%
\item[$\rightarrow$]
}

%Bildquellen
\newcommand*{\quelle}{%
  \tiny Quelle:
}

%Bilder
\newcommand{\bild}[3]{%
	\begin{figure}
	\centering
  \begin{varwidth}{\linewidth}
    \raggedleft
    \includegraphics[scale=#2]{#1}\\
    \quelle\url{#3}
  \end{varwidth}
\end{figure}
}

%Einheiten
\sisetup{
  locale = DE ,
  per-mode = symbol
}

%=====================================================================================================================================
\begin{document}

\frame{\titlepage}
\frame{\frametitle{Gliederung}
\tableofcontents
} %End of frame
%-------------------------------------------------------------------------------------------------------------------------------------
\frame{\frametitle{Einleitendes Beispiel - Mischung von Wasser und Ethanol}
\begin{itemize}
	\item Phänomen: 50ml Wasser + 50ml Ethanol = 96ml Stoffgemisch 
\pause
	\item Zur Erklärung Verwendung eines Modells
\pause
	\item Moleküle = kleine Kügelchen
\pause
	\item Wassermoleküle kleiner als Ethanolmoleküle
\pause
	\item Modellexperiment: Ethanol = Erbsen; Wasser = Reiskörner
\pause
	\item Vermischung der beiden: Kleineres Volumen als die Summe der Einzelvolumina
\pause
	\item Erklärung: Reiskörner füllen die Lücken zwischen den Erbsen
\pause
	\item Schlussfolgerung: Kleinere Wassermoleküle füllen die Lücken zwischen den Ethanolmolekülen
\end{itemize}
} %End of frame
%-------------------------------------------------------------------------------------------------------------------------------------
\section{Eigenschaften von Modellen}
%-------------------------------------------------------------------------------------------------------------------------------------
\frame{\frametitle{Allgemein}
\begin{itemize}
\pause
	\item \textbf{Abstraktion:} Weglassen von Unwichtigem
\pause
	\item \textbf{Dimension:} Vergrößerung / Verkleinerung
\end{itemize}
} %End of frame
%-------------------------------------------------------------------------------------------------------------------------------------
\frame{\frametitle{Anforderungen \& Ziele}
\begin{itemize}
\pause
	\item Abbildung der Wirklichkeit
\pause
	\item Anschaulichkeit
\pause
	\item Einfachheit
\pause
	\item Zeit
\pause
	\item Preis
\pause
	\item Sicherheit
\pause
	\item Beeinflussbarkeit
\end{itemize}
} %End of frame
%-------------------------------------------------------------------------------------------------------------------------------------
\frame{\frametitle{Grenzen}
\pause
\textbf{Beispiel:}
\bild{images/Globus.png}{0.25}{https://pixabay.com/de/globus-erde-welt-karte-geographie-2534766/}
} %End of frame
\frame{\frametitle{Grenzen}
\bild{images/Weltkarte.jpg}{0.1}{https://www.welt-atlas.de/karte_von_welt_politisch_0-9043}
} %End of frame
\frame{\frametitle{Grenzen}
\textbf{Beispiel:}
\bild{images/deutschlandkarte.png}{0.8}{https://www.derweg.org/deutschland/gesamt/deutschlandkarte/}
} %End of frame
\frame{\frametitle{Grenzen}
\textbf{Beispiel:}
\bild{images/berlinKarte.jpg}{0.15}{http://www.city-cover.com/Berlin/city-cover/karte/karte1.htm}
} %End of frame
\frame{\frametitle{Grenzen}
\begin{itemize}
	\item[$\rightarrow$] Jedes Modell hat Grenzen
	\pause
	\item[$\rightarrow$] Modell $\neq$ Wirklichkeit
\end{itemize}
}
%-------------------------------------------------------------------------------------------------------------------------------------
\section{Arten von Modellen}
%-------------------------------------------------------------------------------------------------------------------------------------
\frame{\frametitle{Arten von Modellen}
\centering
%https://tex.stackexchange.com/questions/268578/beamer-uncovering-parts-of-qtree-syntax-tree
\begin{tikzpicture}
  \Tree 
			[.Modelle
					[. \node[visible on=<2->] {Materiell}; \edge [visible on=<3->]; \node[visible on=<3->] {Statisch}; \edge [visible on=<3->]; \node[visible on=<3->] 
							{Simulation}; \edge [visible on=<3->]; \node[visible on=<3->] {\dots}; ]
					[. \node[visible on=<2->] {Ideell}; \edge [visible on=<3->]; \node[visible on=<3->] {Sprachlich}; \edge [visible on=<3->]; \node[visible on=<3->] 
							{Mathematisch}; \edge [visible on=<3->]; \node[visible on=<3->] {\dots}; ]
			];
\end{tikzpicture}
} %End of frame
%-------------------------------------------------------------------------------------------------------------------------------------
\section{Beispiele}
\subsection{Außerhalb der Chemie}
%-------------------------------------------------------------------------------------------------------------------------------------
\frame{\frametitle{Beispiele - Außerhalb der Chemie}
\begin{itemize}
\pause
	\item Technische Modelle
\end{itemize}
\bild{images/TechnischesModell.jpg}{0.5}{https://www.technischesmuseum.at/objekt/modell-eines-dampfsudhauses-1959}
} %End of frame
\frame{\frametitle{Beispiele - Außerhalb der Chemie}
\begin{itemize}
	\item Fotografien:
\end{itemize}
\bild{images/blume.jpg}{0.3}{http://www.matthiashecht.de/index.php?category=Natur&bild=Blume-Natur}
} %End of frame
%-------------------------------------------------------------------------------------------------------------------------------------
\subsection{Chemische Modelle}

\subsubsection{Warum sind Modelle so wichtig?}
%-------------------------------------------------------------------------------------------------------------------------------------
\frame{\frametitle{Warum sind Modelle so wichtig? - Atommodelle}
\begin{itemize}
\pause
	\item \textbf{Man kann Atome nicht sehen.} %Rastertunnelmikroskop -> lediglich elektrische Ströme (sog. Tunnelströme" -> Quantenmechanik)
\pause
	\aitem Modelle zur Erklärung notwendig
\pause
	\item \textbf{Es ist nicht bewiesen, dass es Atome gibt.}
\pause
	\aitem In den Naturwissenschaften kann nichts bewiesen werden!
\end{itemize}
}

\subsubsection{Atommodelle}
%-------------------------------------------------------------------------------------------------------------------------------------
\frame{\frametitle{Atommodell Daltons}
\begin{itemize}
\pause
	\item Materie besteht aus kleinsten, unteilbaren Teilchen $\rightarrow$ Atomen
	\item Atome unterscheiden sich in Masse und Volumen
	\item Elemente bestehen aus einem Atomtyp
	\pause
	\item[$\rightarrow$] Ausreichend für z.B. Brownsche Molekularbewegung
	\pause
	\item[$\rightarrow$] Nicht ausreichend für z.B. Bindungen
\end{itemize}
}

\frame{\frametitle{Atommodell Rutherfords}
\begin{itemize}
	\item[\textbf{$\rightarrow$}] Erklärung der Ergebnisse des Rutherfordschen Streuversuchs
	\pause
	\item Atome sind nicht unteilbar
	\item Kleiner Kern aus Protonen \& Neutronen 
	\item Große fast leere Hülle mit Elektronen auf Kreisbahnen
	\pause
	\item[$\rightarrow$] Ausreichend für z.B. Ionen(bindungen)
	\pause
	\item[$\rightarrow$] Nicht ausreichend für z.B. kovalente Bindungen
	\pause
	\item[$\rightarrow$] \textbf{Warum stürzen Elektronen nicht in den Kern?}
\end{itemize}
}

\frame{\frametitle{Atommodell Bohrs}
\begin{itemize}
	\item Atomhülle besteht aus Schalen $\rightarrow$ Quantenmechanik
	\item Bewegung eines Elektrons nur in Schalen möglich
	\item Kreisbahn in einer Schale stabil
	\pause
	\item[$\rightarrow$] Elektronen stürzen nicht in den Kern
	\pause
	\item[$\rightarrow$] Nicht ausreichend für z.B. kovalente Bindungen
	\pause
	\item[$\rightarrow$] \textbf{Warum hält der Atomkern zusammen?}
\end{itemize}
}

\frame{\frametitle{Orbitalmodell}
\begin{itemize}
	\item Rein mathematisches Modell
	\item Beschreibt den Aufbau der Atomhülle
	\item Umsetzung der Quantenmechanik
	\pause
	\item[$\rightarrow$] Alle Beobachtungen der Chemie erklärbar
\end{itemize}
}

\frame{\frametitle{Fazit}
\begin{itemize}
	\pause
	\item Je detaillierter desto komplexer
	\pause
	\item Modelle entwickeln sich mit der Zeit
\end{itemize}
}
%-------------------------------------------------------------------------------------------------------------------------------------

\end{document}