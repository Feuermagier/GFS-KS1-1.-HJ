\documentclass[a4paper]{article}

\usepackage{ngerman}
\usepackage[ngerman]{babel}
\usepackage[ansinew]{inputenc}
\usepackage[T1]{fontenc}
\usepackage{fancyhdr}
%Baumstrukturen
\usepackage{tikz}
\usepackage{tikz-qtree}
\usepackage{textcomp}   % allows \textrightarrow
%Mathe
\usepackage{amsmath}

\tikzset{
  invisible/.style={opacity=0},
  visible on/.style={alt={#1{}{invisible}}},
  alt/.code args={<#1>#2#3}{%
    \alt<#1>{\pgfkeysalso{#2}}{\pgfkeysalso{#3}}
  }
}

\title{Modelle in der Chemie}
\author{Florian Seligmann}
\date{Dienstag, 12. Dezember 2017}

\pagestyle{fancy}
\lhead{GFS Florian Seligmann}
\rhead{22. Mai 2017}
\chead{}

%----------------------------------------------------------------------------------------------------------
\begin{document}

\begin{center}
\vspace*{\stretch{1.0}}
   \begin{center}
      \Large\textbf{Modelle in der Chemie}\\
      \large\textit{Florian Seligmann}
   \end{center}
\vspace*{\stretch{2.0}}

\section{Eigenschaften}

\begin{itemize}
	\item Abstraktion: Weglassen von Unwichtigem
	\item Dimension: Vergr��erung / Verkleinerung
\end{itemize}

\section{Arten}

\begin{itemize}
	\item Materiell
	\begin{itemize}
		\item Statisch (Kugel-Stab-Modelle, \dots)
		\item Simulation
	\end{itemize}
	\item{Ideell}
	\begin{itemize}
		\item Symbolisch (Reaktionsgleichungen, \dots)
		\item Mathematisch
	\end{itemize}	
\end{itemize}

\section{Anforderungen \& Ziele}

\begin{itemize}
	\item Abbildung der Wirklichkeit
	\item Anschaulichkeit, Einfachheit
	\item Zeit, Preis
	\item Sicherheit, Beeinflussbarkeit
\end{itemize}

\section{Grenzen}
\begin{itemize}
	\item Jedes Modell hat Grenzen
	\item Modell $\neq$ Wirklichkeit
\end{itemize}

\section{Warum sind Modelle so wichtig?}
\begin{itemize}
	\item Chemie arbeitet mit sehr kleinen Dingen $\rightarrow$ Ohne Modelle nicht handhabbar
	\item Naturwissenschaften k�nnen nichts beweisen $\rightarrow$ Alles nur Modelle
\end{itemize}


\end{center}
\end{document}